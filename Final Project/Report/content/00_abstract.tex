\begin{center}

	\makeatletter
	\thispagestyle{plain}
	\LARGE\textbf{\@title} \\
	\vspace{2mm}
	\large\bfseries{\@author} \\
	\normalfont
	\vspace{2mm}
	\large{Winter semester 2018/2019} \\
	\vspace{2mm}
	\large{Lecture Course on Computational Quantum Dynamics \\
		Heidelberg University} \\
	\makeatother
\end{center}

\normalsize

The Gross-Pitaevskii equation is a mean field equation that has been incredibly successful in describing the dynamics of Bose-Einstein condensates. The equation features a non-linear term and thus allows for stable soliton solutions in 1D and additional topological defects such as vortices in 2D. The goal of this project is to study these phenomena using the split-step Fourier method. \\

This report summarizes the results of the final project we conducted as part of the lecture course on Computational Quantum Dynamics, held by Dr. Martin G\"arttner. All numerical work was done using the Python programming language. 

