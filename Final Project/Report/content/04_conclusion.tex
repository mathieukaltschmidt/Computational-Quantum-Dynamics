\section{Conclusion}
In this section we want to conclude our results and compare them to what we would have expected.
\begin{description}
	\item[Solitons in one spatial dimension] 
We analyzed solitonic solutions of the GPE, discretized on a spatial grid. For the case of a black soliton with $\nu=0$ \ we proved, that the distribution does not change over time. At the boundaries of the grid, we could observe some disturbing artefacts, which threaten the stability of our solution after long times.

For a single grey soliton with $\nu = 0.8$ we saw the expected movement in positive $z$-direction. Unfortunately the disturbing effects at the boundaries were a lot stronger in this case. Nevertheless we were able  to reproduce the result one would expect from the analytical solution. The analytical time evolution is also included in the corresponding notebook.
	
In the case of two gray solitons, moving in opposite directions, we saw the overlapping in the center of the grid and after passing past each other the characteristic single movement, not effected by the collision. 
	
	\item[Perturbed density distributions in two spatial dimensions]
We tried to perturb the two dimensional grid on two different ways, first by adding some randomized noise and second by adding a periodic, sinusoidal noise structure. Both distributions showed interesting dynamics over time. The randomized noise seams to vanish after long times, i.\,e. the distribution get's smoother. This was not what we expected intuitively at the beginning , but after playing around with the vortex grids in the third part, it seems that the artificially added defects are not stable at all. The second example showed the same long-term behavior.
  
	\item[Vortices as topological defects]

\end{description}